%!TEX root = ../csuthesis_main.tex
% 设置中文摘要
\keywordscn{天基信息网络;SDN;VNF;资源监测;资源虚拟化}
\categorycn{TP391}
\itemcountcn{图 \totalfigures\ 幅,表 \totaltables\ 个,参考文献 \total{citnum}\ 篇}

\begin{abstractcn}\setlength{\baselineskip}{20pt}%\renewcommand{\baselinestretch}{1.0}
    【研究意义及难点痛点: 天基资源监测和虚拟化是空天地一体化中资源统筹管理和综合应用的重要基础,但目前缺乏有效的资源监测方法和统一的资源表达方式】,【研究思路对策:论文将SDN、NFV分别应用于天基资源监测和虚拟化中】,【具体研究工作:基于 SDN 研究监测子网划分和控制器部署策略,实现根据当前卫星位置和通信情况(口语)调整监测网络,(少了一句)基于NFV 实现天基资源虚拟化,通过软件定义网络的灵活性和虚拟网络功能的可编程性,实现天基资源的虚拟化】,【研究成果:进而实现了一个基于智能体的面向对地观测任务的天基资源管理系统】,【应用效果:  能够高效整合天基资源,协同天基信息网络中各航天器资源执行卫星观测任务】。论文主要的贡献如下:

    (1)基于 SDN 的天基资源监测。【针对问题:为实现天基信息网络监测信息的稳定传输,减少卫星间信息传输延迟】,【思路方法:论文提出了天基信息网络监测子网划分和控制器部署策略以及天基信息网络测量架构】,【研究成果:实现了对全局资源信息监测,保证了信息传输的稳定性、准确性和实时性】。【研究具体过程:在基于SDN构建的天基信息网络中,采用 k-means 算法将天基信息网络划分为不同的子网,在此基础上使用遗传算法确定各个子网中控制器的数量和位置,采用结合主动和被动测量方法的天基信息网络测量架构,【实现结果:达到提高资源监测的全面性和准确性,降低监测对系统性能影响的目的】,【效果证明:实验表明……】。

    (2)基于 NFV 的天基资源虚拟化。【针对问题:针对天基信息网络中资源的异构性以及网络的高度动态性的问题,【思路方法:论文通过软件定义网络的灵活性和虚拟网络功能的可 编程性,实现对天基资源的虚拟化,【研究成果:实现了对异构资源的整合和虚拟网络的生成】。【研究具体过程:论文通过 NFV 实现虚拟网络功能的生成,对虚拟网络功能进行编排以生成虚拟网络,【实现结果:提高了天基信息网络的弹性、可扩展性和资源利用率。】。【效果证明:实验表明……】。

    (3)面向对地观测任务的天基资源管理系统。【针对问题:为实现高效的对地观测】,【运用理论成果:论文利用基于 SDN 的天基资源监测和基于 NFV 的天基资源虚拟化】,【研制系统:研制了面向对地观测任务的天基资源管理系统】【系统实现过程:实现了可以实时监测资源和高效整合天基信息网络资源的系统】,【应用情况:并应用于实际的对地观测场景当中,解决传统系统资源信息反馈不及时、整体资源利用率低的问题】。
\end{abstractcn}