% !TEX program = xelatex
\documentclass{CSUthesis}


\usepackage[figure,table]{totalcount}
\usepackage{totcount}
\usepackage{hyperref}    
%参考文献计数
\newtotcounter{citnum}
\def\oldbibitem{} \let\oldbibitem=\bibitem
\def\bibitem{\stepcounter{citnum}\oldbibitem}
%\total{citnum}
%累积引用次数计数
%\newtotcounter{citesnum}
%\def\oldcite{} \let\oldcite=\cite
%\def\cite{\stepcounter{citesnum}\oldcite}
%\total{citesnum}

%%%%%%%%%%%%%%%%%%%%%%%%%%%%%%%%%%%%%%%%%%%%%%%%%%
% 前置部分的页眉页脚设置
% -----------------------------------------------%
% 正文和后置部分用阿拉伯数字编连续码,前置部分用罗马数字单独编连续码(封面除外)。
% 设置封面页后的页码

% 设置页眉和页脚 
\pagestyle{fancy}
% 正文以前部分无需页眉
\fancyhf{} % 清空原有格式
\renewcommand{\headrulewidth}{0pt}
% 中文摘要之前无需页码

%!TEX root = ../csuthesis_main.tex
% 文章信息
\titlecn{基于SDN/NFV的天基信息网络虚拟化方法}
\titleen{Virtualization Methods for Space-Based Information Networks Based on SDN/NFV \\Central South University}

\priormajor{工程硕士}
\minormajor{电子信息}
\interestmajor{计算机技术}
\author{李启茂}
\supervisor{费洪晓}
\subsupervisor{}
\department{计算机学院}
\studentid{224712126}
\thesisdate{year=2025,month=4}



\clcnumber{TP391} 				% 中图分类号 Chinese Library Classification
\schoolcode{10533}			% 学校代码
\udc{004.9}						% UDC
\academiccategory{专业学位}	% 学术类别


\newif \ifblindreview % 条件语句,是否是盲审版本
\newif \ifAcademic % true为学术学位,false为专业学位
\newif \ifPublic   % true为公开,false为涉密
\newif \ifDoctor   % 是否为博士学位论文



%\blindreviewtrue  %%盲审版本
\blindreviewfalse  %%正常版本

%\Academictrue % 学术学位
\Academicfalse % 专业学位

\Publictrue  % 公开
%\Publicfalse % 涉密

%\Doctortrue  % 博士学位论文
\Doctorfalse % 硕士学位论文




\begin{document}
%%%%%%%%%%%%%%%%%%%%%%%%%%%%%%%%%%%%%%%%%%%%%%%%%%
% 封面
% -----------------------------------------------%
\makecoverpage
\clearpage
\mbox{}
\clearpage

\ifblindreview	% 盲审不需要扉页和声明页
\else
	%%%%%%%%%%%%%%%%%%%%%%%%%%%%%%%%%%%%%%%%%%%%%%%%%%
	% 扉页 
	% -----------------------------------------------%

	\maketitlepage

	\clearpage
	\mbox{}
	\clearpage

	\setlength{\headheight}{9.6mm}
	\setlength{\footskip}{7.9mm}
	\setlength{\textheight}{681.5pt} % 调整版面高度
	%%%%%%%%%%%%%%%%%%%%%%%%%%%%%%%%%%%%%%%%%%%%%%%%%%
	% 声明页
	% -----------------------------------------------%
	\announcement
	\clearpage
	\mbox{}
	\clearpage
\fi
%%%%%%%%%%%%%%%%%%%%%%%%%%%%%%%%%%%%%%%%%%%%%%%%%%

% 设置页眉和页脚 %
\setlength{\headheight}{9.6mm}
\setlength{\footskip}{7.9mm}
\pagestyle{fancy}
\fancyhf[CF]{\TimesNewRoman \zihao{-5} \thepage} % 所有(奇数和偶数)中间页脚,TimesNewRoman小五号

\addtocontents{toc}{\vspace{-8pt}}

% 中文摘要
% -----------------------------------------------%
\pagenumbering{Roman} % 重摘要至目录部分页码为大写罗马字体
\setcounter{page}{1} % 页码从I重新开始
\phantomsection                %% 解决目录中超链接地址错误问题
\addcontentsline{toc}{section}{摘要} %%增加摘要至目录且与第一章对齐
%!TEX root = ../csuthesis_main.tex
% 设置中文摘要
\keywordscn{天基信息网络;SDN;VNF;资源监测;资源虚拟化}
\categorycn{TP391}
\itemcountcn{图 \totalfigures\ 幅,表 \totaltables\ 个,参考文献 \total{citnum}\ 篇}

\begin{abstractcn}\setlength{\baselineskip}{20pt}%\renewcommand{\baselinestretch}{1.0}
    【研究意义及难点痛点: 天基资源监测和虚拟化是空天地一体化中资源统筹管理和综合应用的重要基础,但目前缺乏有效的资源监测方法和统一的资源表达方式】,【研究思路对策:论文将SDN、NFV分别应用于天基资源监测和虚拟化中】,【具体研究工作:基于 SDN 研究监测子网划分和控制器部署策略,实现根据当前卫星位置和通信情况(口语)调整监测网络,(少了一句)基于NFV 实现天基资源虚拟化,通过软件定义网络的灵活性和虚拟网络功能的可编程性,实现天基资源的虚拟化】,【研究成果:进而实现了一个基于智能体的面向对地观测任务的天基资源管理系统】,【应用效果:  能够高效整合天基资源,协同天基信息网络中各航天器资源执行卫星观测任务】。论文主要的贡献如下:

    (1)基于 SDN 的天基资源监测。【针对问题:为实现天基信息网络监测信息的稳定传输,减少卫星间信息传输延迟】,【思路方法:论文提出了天基信息网络监测子网划分和控制器部署策略以及天基信息网络测量架构】,【研究成果:实现了对全局资源信息监测,保证了信息传输的稳定性、准确性和实时性】。【研究具体过程:在基于SDN构建的天基信息网络中,采用 k-means 算法将天基信息网络划分为不同的子网,在此基础上使用遗传算法确定各个子网中控制器的数量和位置,采用结合主动和被动测量方法的天基信息网络测量架构,【实现结果:达到提高资源监测的全面性和准确性,降低监测对系统性能影响的目的】,【效果证明:实验表明……】。

    (2)基于 NFV 的天基资源虚拟化。【针对问题:针对天基信息网络中资源的异构性以及网络的高度动态性的问题,【思路方法:论文通过软件定义网络的灵活性和虚拟网络功能的可 编程性,实现对天基资源的虚拟化,【研究成果:实现了对异构资源的整合和虚拟网络的生成】。【研究具体过程:论文通过 NFV 实现虚拟网络功能的生成,对虚拟网络功能进行编排以生成虚拟网络,【实现结果:提高了天基信息网络的弹性、可扩展性和资源利用率。】。【效果证明:实验表明……】。

    (3)面向对地观测任务的天基资源管理系统。【针对问题:为实现高效的对地观测】,【运用理论成果:论文利用基于 SDN 的天基资源监测和基于 NFV 的天基资源虚拟化】,【研制系统:研制了面向对地观测任务的天基资源管理系统】【系统实现过程:实现了可以实时监测资源和高效整合天基信息网络资源的系统】,【应用情况:并应用于实际的对地观测场景当中,解决传统系统资源信息反馈不及时、整体资源利用率低的问题】。
\end{abstractcn}
\clearpage

%%%%%%%%%%%%%%%%%%%%%%%%%%%%%%%%%%%%%%%%%%%%%%%%%%
% 英文摘要
% -----------------------------------------------%
\phantomsection                %% 解决目录中超链接地址错误问题
\addcontentsline{toc}{section}{ABSTRACT}
\include{content/abstracten}
\clearpage

% 目录
% -------------------------------------------%
{
	\renewcommand*{\baselinestretch}{1.3841}   % 行间距20pt
	\renewcommand{\contentsname}{\vspace{10pt} \hfill \heiti \bfseries \zihao{3} 目\quad 录\hfill ~\\}
	\phantomsection                %% 解决目录中超链接地址错误问题
	\addcontentsline{toc}{section}{目录} % 在目录中添加目录页码.
	\tableofcontents
	\label{test}
}
\clearpage

% 插图索引
% -------------------------------------------%
{
	\setlength{\baselineskip}{20pt}         % 基准行间距
	\renewcommand{\baselinestretch}{1.0}   % 几倍行间距
	{~}
	\vspace{-10pt}
	\renewcommand{\listfigurename}{\hfill \heiti \zihao{3} 插图索引\hfill ~\\}
	% 在目录中添加插图索引页码,按需确定是否注释下面两行
	\phantomsection                %% 解决目录中超链接地址错误问题
	\addcontentsline{toc}{section}{插图索引}
	%%设置插图索引的 图 标签
	\let\oldnumberline\numberline%
	\renewcommand{\numberline}{\figurename~\oldnumberline}
	%增加图目录,按需确定是否注释下面一行
	\listoffigures
}

\clearpage
% 表格索引
% -------------------------------------------%
{
	\setlength{\baselineskip}{20pt}         % 基准行间距
	\renewcommand{\baselinestretch}{1.0}   % 几倍行间距
	{~}
	\vspace{-10pt}
	\renewcommand{\listtablename}{\hfill \heiti \zihao{3} 表格索引\hfill ~\\}
	% 在目录中添加表格索引页码,按需确定是否注释下面两行
	\phantomsection                %% 解决目录中超链接地址错误问题
	\addcontentsline{toc}{section}{表格索引}
	%%设置表格索引的 表 标签
	\let\oldnumberline\numberline%
	\renewcommand{\numberline}{\tablename~\oldnumberline}
	%增加表目录,按需确定是否注释下面一行
	\listoftables
}

\clearpage


%%%%%%%%%%%%%%%%%%%%%%%%%%%%%%%%%%%%%%%%%%%%%%%%%%
% 符号说明(必要时)
% -----------------------------------------------%
\phantomsection                %% 解决目录中超链接地址错误问题
\addcontentsline{toc}{section}{符号说明} % 在目录中添加目录页码.
\include{content/denotation}

%%%%%%%%%%%%%%%%%%%%%%%%%%%%%%%%%%%%%%%%%%%%%%%%%%
% 正文页眉页脚
% -----------------------------------------------%

\renewcommand{\headrulewidth}{1pt}

% 去掉页眉章节序号后面的“.”
\renewcommand{\sectionmark}[1]{\markboth{第{\thesection}章~ #1}{第{\thesection}章~ #1}}
\renewcommand{\subsectionmark}[1]{\markright{\leftmark}}
\renewcommand{\subsubsectionmark}[1]{\markright{\leftmark}}

\fancyhf[RH]{\songti \zihao{5} \leftmark} % 设置所有(奇数和偶数)右侧页眉net
\ifDoctor
	\fancyhf[LH]{\songti \zihao{5} 中南大学博士学位论文,第\pageref{test}页} % 设置所有(奇数和偶数)左侧页眉  
\else
	\fancyhf[LH]{\songti \zihao{5} 中南大学硕士学位论文,第\pageref{test}页} % 设置所有(奇数和偶数)左侧页眉  
\fi
% 正文内容 
% --------------------------------------------%
\setcounter{page}{1} % 重置页码编号
\pagenumbering{arabic} % 设置页码编号为阿拉伯数字

% 可以使用include命令导入tex文件,从而避免过多修改本文件。

% 论文正文是主体,主体部分应从另页右页开始,每一章应另起页。一般由序号标题、文字叙述、图、表格和公式等五个部分构成。

% 重新设置正文行间距,因为前置部分设置时候行间距被改过
\renewcommand*{\baselinestretch}{1.0}   % 几倍行间距
\setlength{\baselineskip}{20pt}         % 基准行间距

% 正文
{
	% 表格字号应比正文小,一般五号/10.5pt,但是暂时没法再cls里设置(不然会影响到封面等tabular环境)
	% 所以目前只好在主文件里局部\AtBeginEnvironment
	\AtBeginEnvironment{tabular}{\zihao{5}}
	\include{content/content}
}

%%%%%%%%%%%%%%%%%%%%%%%%%%%%%%%%%%%%%%%%%%%%%%%%%%
% 临时标签,用于编译时追踪正文末尾
%%%%%%%%%%%%%%%%%%%%%%%%%%%%%%%%%%%%%%%%%%%%%%%%%%

%%%%%%%%%%%%%%%%%%%%%%%%%%%%%%%%%%%%%%%%%%%%%%%%%%
% 后续内容,标题三号黑体居中,章节无编号
% --------------------------------------------%

% https://www.zhihu.com/question/29413517/answer/44358389 %
% 说明如下:
% secnumdepth 这个计数器是 LaTeX 标准文档类用来控制章节编号深度的。在 article 中,这个计数器的值默认是 3,对应的章节命令是 \subsubsection。也就是说,默认情况下,article 将会对 \subsubsection 及其之上的所有章节标题进行编号,也就是 \part, \section, \subsection, \subsubsection。LaTeX 标准文档类中,最大的标题是 \part。它在 book 和 report 类中的层级是「-1」,在 article 类中的层级是「0」。这里,我们在调用 \appendix 的时候将计数器设置为 -2,因此所有的章节命令都不会编号了。不过,一般还是会保留 \part 的编号的。所以在实际使用中,将它设置为 0 就可以了。

% 在修改过程中请注意不要破环命令的完整性

\renewcommand\appendix{\setcounter{secnumdepth}{-2}}
\appendix

% 主文件有代码去掉页眉章节编号的“.”,但这会因为bug导致无编号章节显示一个错误编号,所以这里在无编号章节之前再次重定义sectionmark。
\renewcommand{\sectionmark}[1]{\markboth{#1}{#1}}
\renewcommand{\subsectionmark}[1]{\markright{\leftmark}}
\renewcommand{\subsubsectionmark}[1]{\markright{\leftmark}}

% section 标题从这里往后改为三号黑体居中
\titleformat{\section}{\centering \zihao{3} \bfseries \heiti}{\thesection}{1em}{}
\titlespacing*{\section} {0pt}{18pt}{32pt}

% \section{参考文献} % bibliography会自动显示参考文献四个字

% \nocite{*} % 该命令用于显示全部参考文献,即使文中没引用
% cls文件中已经引入package,这里不需要调用 \bibliographystyle 了。
% \bibliographystyle{gbt7714-2005} 
{~}
\vspace{18pt}
\phantomsection                                      %% 解决目录中超链接地址错误问题
\addcontentsline{toc}{section}{参考文献} % 由于参考文献不是section,这句把参考文献加入目录
\zihao{-4} \setmainfont{Times New Roman}
\setlength{\baselineskip}{20pt}         % 基准行间距 
\bibliographystyle{gbt7714-2015-little}
\bibliography{thesis-references}

\clearpage
% 附录

\titleformat{\section}{\centering \zihao{3} \bfseries \heiti}{\thesection}{1em}{}
\titlespacing*{\section} {0pt}{18pt}{32pt}
\titleformat{\subsection}{\zihao{-4} \songti}{\thesubsection}{1em}{}
\titlespacing*{\subsection} {2em}{0pt}{0pt}
\titleformat{\subsubsection}{\zihao{-4} \songti}{\thesubsubsection}{1em}{}
\titlespacing*{\subsubsection} {2em}{0pt}{0pt}

\include{content/appendix}

\clearpage
% 研究成果和致谢

\titleformat{\section}{\centering \zihao{3} \bfseries \heiti}{\thesection}{1em}{}
\titlespacing*{\section} {0pt}{18pt}{32pt}
\titleformat{\subsection}{\zihao{-4} \songti}{\thesubsection}{1em}{}
\titlespacing*{\subsection} {2em}{0pt}{0pt}
\titleformat{\subsubsection}{\zihao{-4} \songti}{\thesubsubsection}{1em}{}
\titlespacing*{\subsubsection} {2em}{0pt}{0pt}

\include{content/additional}


\end{document}
